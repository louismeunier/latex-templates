\documentclass[12pt]{article}
\usepackage{amsthm}
\usepackage{libertine}
\usepackage[margin=0.15in]{geometry}
\usepackage{amsmath,amssymb}
\usepackage{multicol}
\usepackage[shortlabels]{enumitem}
\usepackage{siunitx}
\usepackage{setspace}
\usepackage{cancel}
\usepackage{graphicx}
\usepackage{pgfplots}
\usepackage{listings}
\usepackage{tabularx}
\usepackage{titlesec}
\usepackage{thmtools}
\usepackage{thm-restate}
\usepackage[colorlinks=true, linkcolor=darkgray]{hyperref}
\usepackage{cleveref}
\usepackage[]{csquotes}

% makes theorems, definitions, etc. "restatable" as shown
% can add more with same format as you wish
\declaretheorem[name=Definition,numberwithin=section]{definition}
\declaretheorem[name=Lemma,numberwithin=section]{lemma}
\declaretheorem[name=Theorem,numberwithin=section]{theorem}

% makes "quoted" text actually look correct
\MakeOuterQuote{"}

% page footer
\newpagestyle{mypage}{%
    \footrule
    \setfoot{\small\textcolor{gray}{§\thesubsection}}{\small\textcolor{gray}{\textit{\sectiontitle: \textbf{\subsectiontitle}}}}{\textcolor{gray}{\small p. \thepage}}
}


% tikz/pgfplot settings
\pgfplotsset{width=10cm,compat=1.9}
\usepgfplotslibrary{external}
% \tikzexternalize[shell escape=-enable-write18]

% title page settings
\newcommand{\pageauthor}{Louis Meunier}
\newcommand{\pagetitle}{Intro To Very Special Math}
\newcommand{\pagesubtitle}{MATH123}

% black square for qed symbol
\renewcommand{\qedsymbol}{$\blacksquare$}

\titleformat{\section}
{\centering\normalfont\Large\bfseries}
{\thesection}{1em}{}

\begin{document}
\setstretch{2.25}
\noindent
\begin{center}
    \begin{tabularx}{\textwidth} { 
        >{\raggedright\arraybackslash}X 
        >{\raggedleft\arraybackslash}X}
    \LARGE \pageauthor \\
    \LARGE \textbf{\pagetitle} & \LARGE \textbf{\pagesubtitle}\\
    \end{tabularx}\\
    \rule[2ex]{0.8\textwidth}{1pt}
\end{center}

\setstretch{1.5}
\tableofcontents

% "enables" footer with section+subsection, etc. just comment it out if you don't want it
\pagestyle{mypage}

% makes sections a very dark gray + centered
\titleformat{\section}
{\color{darkgray}\centering\normalfont\Large\bfseries}
{\color{darkgray}\thesection}{1em}{}

% need to change margins and such here for rest of document
% kind of messy but what can you do
\newpage
% modify these as you wish
\newgeometry{margin=0.5in, top=0.4in, bottom=0.75in}
\parskip=0.5em

\section{Some section}
\subsection{Some subsection}
\subsubsection{Some subsubsection}
\begin{definition}[Composite]
  Given $f: A \rightarrow B$ and $g: B \rightarrow C$, the \textbf{composite} $g \circ f$ of $f$ and $g$ is the function $g \circ f: A \rightarrow C$, defined by the equation $(g\circ f)(a) = g(f(a))$. The rule of this composition: \[\{(a,c)\,|\,\text{For some } b \in B, f(a) = b \text{ and } g(b) = c\}\]
\end{definition}

Using the si-unit package, big numbers look pretty nice, \num{1e-10}.
\begin{restatable}[]{lemma}{lemmaequivalence}
  \label{lemma:equivalence}
  Two equivalence classes are either disjoint or equal.
\end{restatable}

\section{Appendix}

\lemmaequivalence*
\begin{proof}[Proof of \Cref{lemma:equivalence}]
  Let $E$ be the equivalence class determined by $x$ and $E'$ the equivalence class determined by $x'$. Assume $E \cap E' \neq \varnothing$, and take $y \in E \cap E'$. By definition of equivalence classes, $y \sim x$ and $y \sim x'$. 
  
  By the property of symmetry, $x \sim y$, and by transivity, $x \sim x'$. Thus, $x \in E'$, and it follows that since $x \in E$, $E \subset E'$. The same logic applies in the other direction, such that $E' \subset E$, and thus $E = E'$.
\end{proof}
\end{document}