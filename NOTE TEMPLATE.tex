\documentclass[12pt,oneside]{article}
\usepackage{amsthm}
\usepackage{libertine}
\usepackage[margin=0.15in]{geometry}
\usepackage{amsmath,amssymb}
\usepackage{multicol}
\usepackage[shortlabels]{enumitem}
\usepackage{siunitx}
\usepackage{setspace}
\usepackage{cancel}
\usepackage{graphicx}
\usepackage{pgfplots}
\usepackage{listings}
\usepackage{tabularx}
\usepackage{titlesec}
\usepackage{thmtools}
\usepackage{thm-restate}
\usepackage{xcolor-solarized}
\usepackage[side, ragged]{footmisc}
\usepackage{marginnote}
\usepackage{xsim}
\usepackage[colorlinks=true, linkcolor=darkgray]{hyperref}
\usepackage[raggedrightboxes]{ragged2e}
\usepackage{cleveref}
\usepackage[]{csquotes}
% \usepackage{shortcuts}

% can add more with same format as you wish
\renewcommand*{\qedsymbol}{\(\blacksquare\)}
% homework answers
% TODO: uncomment to show answers
% optional: use "problem" and corresponding "solution" environments. solutions are hidden when solution/print = true
\xsimsetup{
  load-style=layouts,
  % solution/print=true,
  exercise/template=minimal,
  solution/template=runin,
}
% TODO: subsection numbering
\declaretheorem[
  % thmbox=S,
  name=Definition,
  refname={Definition, definition}, numberwithin=subsection,
  shaded={rulecolor=solarized-blue, rulewidth=2pt}
]{definition}

\declaretheorem[
  % thmbox=S,
  name=Axiom,
  refname={Axiom, axiom},
  numberwithin=section,
  shaded={rulecolor=solarized-orange, rulewidth=2pt}
]{axiom}

\declaretheorem[
  % thmbox=S,
  name=Lemma,
  refname={Lemma, lemma},
  numberwithin=section,
  shaded={rulecolor=solarized-orange, rulewidth=1pt, bgcolor={rgb}{1,1,1}}
]{lemma}

\declaretheorem[
  % thmbox=S,
  name=Corollary,
  refname={Corollary, corollary},
  numberwithin=section,
  shaded={rulecolor=solarized-orange, rulewidth=1pt, bgcolor={rgb}{1,1,1}}
]{corollary}

\declaretheorem[
  % thmbox=S,
  name=Remark,
  refname={Remark, remark},
  numberwithin=section
]{remark}

\declaretheorem[
  % thmbox=S,
  name=Theorem,
  refname={Theorem, theorem},
  numberwithin=section,
  shaded={rulecolor=solarized-red, rulewidth=2pt}
]{theorem}

\declaretheorem[
  % thmbox=M,
  name=Example,
  refname={Example, example},
  numberwithin=section,
  shaded={rulecolor=solarized-cyan, rulewidth=1pt, bgcolor={rgb}{1,1,1}}
]{example}

\declaretheorem[
  % thmbox=S,
  name=Proposition,
  refname={Proposition, proposition},
  numberwithin=section,
  shaded={rulecolor=solarized-magenta, rulewidth=1pt, bgcolor={rgb}{1,1,1}}
]{proposition}

% makes "quoted" text actually look correct
\MakeOuterQuote{"}

% page footer
\newpagestyle{mypage}{%
    \footrule
    \setfoot{\small\textcolor{gray}{§\thesubsection}}{\small\textcolor{gray}{\textit{\sectiontitle: \textbf{\subsectiontitle}}}}{\textcolor{gray}{\small p. \thepage}}
}

% title page settings
\newcommand{\pageauthor}{Some One}
\newcommand{\pagetitle}{Course Name}
\newcommand{\pagesubtitle}{MATH123}

% black square for qed symbol
\renewcommand{\qedsymbol}{$\blacksquare$}

\titleformat{\section}
{\centering\normalfont\Large\bfseries}
{\thesection}{1em}{}

\begin{document}
\setstretch{2.25}
\noindent
\begin{center}
    \begin{tabularx}{\textwidth} { 
        >{\raggedright\arraybackslash}X 
        >{\raggedleft\arraybackslash}X}
    \LARGE \pageauthor \\
    \LARGE \textbf{\pagetitle} & \LARGE \textbf{\pagesubtitle}\\
    \end{tabularx}\\
    \rule[2ex]{0.8\textwidth}{1pt}
\end{center}

\setstretch{1.5}
\tableofcontents

% "enables" footer with section+subsection, etc. just comment it out if you don't want it
\pagestyle{mypage}

% makes sections a very dark gray + centered
\titleformat{\section}
{\color{darkgray}\centering\normalfont\Large\bfseries}
{\color{darkgray}\thesection}{1em}{}

% need to change margins and such here for rest of document
% kind of messy but what can you do
\newpage
% modify these as you wish
\newgeometry{margin=0.25in, top=0.4in, bottom=0.5in, marginparwidth=1.4in, marginparsep=0.3in, outer=0.2in, includemp}
\parskip=0.6em

\section{Some section}
\subsection{Some subsection}
\subsubsection{Some subsubsection}
\begin{remark}
  Some remark; ah yes.
\end{remark}

\begin{theorem}\label{thm:1}
  Colorful theorems are cool.
\end{theorem}

\begin{axiom}
  Some axiom.
\end{axiom}

\begin{example}
  % crefs are cool
  Prove \Cref{thm:1}.
  \begin{proof}
    Some proof.
  \end{proof}
\end{example}

\begin{definition}[Some definition]
  Some definition of a thing.\footnotemark
\end{definition}
\footnotetext{Make a note here (a footnote, technically, but in the margin)}
% * unfortunately, \footnote{} does not work in float environments, so you need to use the above trick

\begin{proposition}
  Some proposition.
  \begin{proof}
    Some proof.
  \end{proof}
\end{proposition}

\begin{lemma}
  Some lemma.
\end{lemma}

\begin{corollary}
  Some corollary.
\end{corollary}

\end{document}